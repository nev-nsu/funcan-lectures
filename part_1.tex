\documentclass[12pt]{article}

\usepackage{cmap}
\usepackage{amssymb}
\usepackage[T2A]{fontenc}
\usepackage[utf8]{inputenc}
\usepackage[russian]{babel}
\usepackage{amsmath}
\usepackage{graphicx}
\usepackage{dcolumn}
\usepackage{hyperref}
\usepackage{stackrel}
\usepackage{amsmath}
\usepackage{MnSymbol}
\usepackage{wasysym}
\usepackage{geometry}
\geometry{verbose,tmargin=3cm,bmargin=3cm,lmargin=3cm,rmargin=3cm}
\usepackage{setspace}
\newcommand{\csp}{\ensuremath{\mathbb{C}}}
\newcommand{\real}{\ensuremath{\mathbb{R}}}
\newcommand{\dd}{\ensuremath{\mathrm{d}}}
\setcounter{tocdepth}{2}
\hypersetup{
    colorlinks=true, %set true if you want colored links
    linktoc=all,     %set to all if you want both sections and subsections linked
    linkcolor=blue,  %choose some color if you want links to stand out
}
\title{Ряды Фурье}

\begin{document}

\tableofcontents
\newpage

\section{Ряды Фурье}

\subsection{Коэффициенты Фурье и ряд Фурье}

\textbf{Определение.} Тригонометрический многочлен -- это линейная комбинация функций $\sin{nx}$ и $\cos{nx}, n \in \mathbb{N}$.

\textbf{Определение.} Тригонометрический ряд Фурье -- это ряд вида $$\frac{a_0}{2} + \sum_{n=1}^\infty  \big(a_n \cos{nx} + b_n \sin{nx}\big).$$

Будем рассматривать ряды, для которых выполнено условие $$\sum_{n=1}^\infty \big( |a_n| + |b_n| \big) < \infty.$$ В этом случае ряд Фурье сходится.

\paragraph{}

\textit{Упражнение.} Докажите, что если функция периодическая, то интеграл по периоду не зависит от места интегрирования.

\paragraph{}

Рассмотрим следующие интегралы:

$$\int_{-\pi}^{\pi} \cos{nx} \cos{kx} \mathrm{d}x = \int_{-\pi}^{\pi} \frac{\cos{(n-k)x} + \cos{(n+k)x}}{2} \mathrm{d}x = \begin{cases}
    0,   & n \ne k, \\
    \pi, & n =  k \\
\end{cases}$$


$$\int_{-\pi}^{\pi} \sin{nx} \sin{kx} \mathrm{d}x = \int_{-\pi}^{\pi} \frac{\cos{(n-k)x} - \cos{(n+k)x}}{2} \mathrm{d}x = \begin{cases}
    0,   & n \ne k, \\
    \pi, & n =  k \\
\end{cases}$$

$$\int_{-\pi}^{\pi} \cos{nx} \sin{kx} \mathrm{d}x = 0 $$

Заметим, что если определить на пространстве функций скалярное произведение как $$(f, g) = \int_{-\pi}^{\pi} f(x) g(x) \mathrm{d}x, $$ то синусы оказываются ортогональны косинусам. 

\paragraph{}

Пусть известно $f(x)$. Найдем коэффициенты разложения:

\paragraph{}

\textit{Случай 1.} $n \ne 0$:

$$\int_{-\pi}^{\pi} f(x) \sin{nx} \mathrm{d}x = \int_{-\pi}^{\pi}\big( \frac{a_0}{2} + \sum_{k \geq 1} \big( a_k\cos{kx} + b_k \sin{kx} \big) \big) \sin{nx} \mathrm{d}x = ...$$
Промежуток интегрирования ограниченный, сходимость ряда равномерная, следовательно, можно менять местами интеграл и суммирование.
$$... = \frac{a_0}{2} \int \sin{nx} \dd x + \sum \int a_k\cos{kx}\sin{nx}\dd x + \sum \int b_k\sin{kx}\sin{nx}\dd x = b_n\pi .$$


\begin{align*}
    & \text{Аналогично предыдущему}, \\
    & \int_{-\pi}^{\pi} f(x) \cos{nx}\dd x = \\ & \int_{-\pi}^{\pi}\big(\frac{a_0}{2} + \sum a_k\cos{kx} + b_k\sin{kx}\big)\cos{nx}\dd x = \\  
    \frac{a_0}{2}&\int\sin{nx}\dd x + \sum\int a_k\cos{kx}\sin{nx}\dd x + \sum\int b_k\sin{kx}\sin{nx}\dd x = a_n\pi.
\end{align*}

\paragraph{}

\textit{Случай 2.} $n = 0$:

$$\int_{-\pi}^{\pi} f(x) \cos{nx} = \pi a_0 $$



В результате имеем: 
\begin{itemize}
    \item $a_n = \frac{1}{\pi} \int_{-\pi}^{\pi} f(x) \cos{nx} \dd x $
    \item $b_n = \frac{1}{\pi} \int_{-\pi}^{\pi} f(x) \sin{nx} \dd x $
\end{itemize}

Найденный нами коэффициенты называются \textit{коэффициентами Эйлера--Фурье}.

\paragraph{}

Достаточное условие сходимости ряда Фурье: $|(f(x)|$ интегрируема от $-\pi$ до $\pi$.

\textbf{Лемма.} Пусть $f_n \rightrightarrows f$.
Если есть интегрируемая мажоранта, то предельная функция будет интегрируема и $$\int f(x) \dd x = \lim \int f_n(x) \dd x. $$

\paragraph{}

Пусть $f \in L_1(\mathrm{R})$, где $L_1$ - это векторное пространство интегрируемых функций.

Определим теперь на пространстве $L_1$ норму. Будем считать одним элементом множества класс функций, которые отличаются на множество меры ноль.
$$||f|| = \int_\mathrm{R} |f|$$

После введения нормы мы можем определить равномерную сходимость:
$$f_n \rightrightarrows f, \quad \text{если} \quad ||f_n - f|| \rightarrow 0$$

\paragraph{}

\textbf{Определение} Ступенчатая функция - это конечная линейная комбинция индикаторов интервалов.

\paragraph{}

\textbf{Теорема} Для любого $\varepsilon > 0$ существует ступенчатая функция $h$ такая, что $||f - h|| < \varepsilon$.
(Любую функцию можно с бесконечной точностью приблизить к ступенчатой).

\vspace{1em}

\textbf{Лемма} (Римана-Лебега)

Если $f \in L_1(\mathrm{R})$, то $$\lim_{\lambda \to \infty} \int_{-\infty}^{\infty} f(x) \sin {\lambda x} = 0.$$

\textbf{Доказательство}. \textit{Тут должно было быть доказательство, но оно потерялось. См. \href{http://old.nsu.ru/education/funcan/node39.html}{тут}.}

\newpage
\subsection {Ядро Дирихле}

\textbf{Определение.} Ядром Дирихле называется следующая функция:

$$ D_n(a) = \frac{1}{2} + \sum_{k=1}^{n} \cos{ak} = \frac{\sin {\big(\frac{2n+1}{2} a \big)}}{2 \sin {\frac{a}{2}}}.$$

$D_n$ - чётная, 2$\pi$-периодическая функция, интеграл которой по периоду равен $\pi$.

\vspace{1em}

Найдём частичную сумму ряда Фурье:
 \begin{align*}
    S_n (x) = & \frac{a_0}{2} + \sum_{k=1}^{n} \big( a_k \cos{kx} + b_k \sin{kx} \big) = \\
              & \frac{1}{2} \big( \frac{1}{\pi} \int_{-\pi}^{\pi} f(u) \dd u \big) + \sum_{k} \big( \big( \frac{1}{\pi} \int_{-\pi}^{\pi} f(u)\cos{ku} \dd u \big) \cos{kx} 
              + \big( \frac{1}{\pi} \int_{-\pi}^{\pi} f(u) \sin{ku} \dd u \big) \sin{kx} \big) = \\ 
              & \frac{1}{\pi} \int f(u) \big( \frac{1}{2} + \sum (\sin{ku}\sin{kx} + \cos{ku}\cos{kx}) \big) \dd u = \\
              & \frac{1}{\pi} \int f(u) \big( \frac{1}{2} + \sum\cos(a-x) \big) \dd u = \frac{1}{\pi} \int f(u) D_n(u-x) \dd u \stackbin{t = u - x}{=} \\
              & \frac{1}{\pi} \int_{-\pi}^{\pi} f(t+x) D_n (t) \dd t = \frac{1}{\pi} \int_{0}^{\pi} f(x - t) D_n(t) \dd t + \frac{1}{\pi} \int_{0}^{\pi} f(x+t) D_n(t) \dd t = \\
              & \frac{1}{\pi} \int_{0}^{\pi} \big( f(x+t) + f(x-t) \big)) D_n(t) \dd t
          \end{align*}
  
  \subsection{Принцип локализации}
  
  
  Пусть $\pi > \delta > 0, \delta \rightarrow 0$.
  
  $$S_n = \frac{1}{\pi} \big( \int_{0}^{\delta} \dots + \int_{\delta}^{\pi} \dots \big) = 
  \frac{1}{\pi} \int_{0}^{\delta}\dots + \frac{1}{\pi} \int_{\delta}^{\pi} (f(x+t)+f(x-t)) \frac{\sin{\frac{2n+1}{2}t}}{2\sin{\frac{t}{2}}} \dd t$$
  
  Воспользуемся леммой Римана-Лебега: второе слагаемое стремится к нулю при $n \rightarrow \infty$. Первое же слагаемое зависит только от поведения функции в $\delta$-окрестности.

  \paragraph{} 

  Таким образом, мы доказали:
  
  \textbf{Теорема} (Римана). Если две функции совпадают в $\delta$-окрестности точки $x$, то их ряды Фурье сходятся или расходятся одноверменно. Если их ряды Фурье сходятся,
  то суммы рядов равны.
  
  При этом ряды могут кардинально отличаться друг от друга на всём промежутке.
  
 \newpage

\subsection{Условие представимости функции рядом Фурье}

 \textbf{Теорема.} Для того, чтобы функция совпадала с суммой своего ряда Фурье на всей оси, достаточны:
  
 \begin{itemize}
     \item \textit{Для существования ряда Фурье: интегрируемость}
     \item $2\pi$-периодичность
     \item непрерывность
         \begin{itemize}
             \item функция имеет на периоде конечное число точек разрыва
             \item все разрывы только первого рода
             \item $f(x) = \frac{f(x + 0) + f(x - 0)}{2}$ ~ в точке разрыва регулярное значение
        \end{itemize}         
    \item дифференцируемость
        \begin{itemize}
            \item конечное число точек недифференцируемости
            \item производная ограничена
         \end{itemize}
\end{itemize}

  \paragraph{}
  
  \textbf{Доказательство.}

  
  $$S_n(x) - A = \frac{1}{\pi} \int_{0}^{\pi} \big( f(x+t)+f(x-t) - 2A \big) D_n \dd t$$
  
  \begin{align*}
      S_n(x) - f(x) & = \frac{1}{\pi} \int_{0}^{\pi} \big( f(x+t)+f(x-t)-f(x+0)-f(x-0) \big) D_n(t) \dd t = \\
      & = \frac{1}{\pi} \int_{0}^{\delta} (f(x+t)+f(x-t)-f(x+0)-f(x-0))D_n(t) \dd t + o(1) = \\ 
      & = \frac{1}{\pi} \big( \int_{0}^{\delta} (f(x+t)-f(x+0))\frac{\sin(\frac{2n+1}{2}t)}{2\sin{\frac{t}{2}}} + \\
      & \qquad{} + \int_{0}^{\delta} \left( f(x+t) - f(x+0) \right) D_n(t) \dd t \big) + o(1) =
  \end{align*}
  
  Рассмотрим только первый интеграл (для второго рассуждения аналогичны). Докажем, что он стремится к нулю.

\newpage

  Рассмотрим функцию  $$ \frac{f(x+t)-f(x+0)}{2\sin{\frac{t}{2}}}.$$ Выберем $\delta$ так, 
  чтобы на промежутках $(x, x + \delta]$ и $[x - \delta, x)$ у этой функции не было точек разрыва и недифференцируемости.
  В силу необходимых условий это возможно.
 
Воспользуемся теоремой Лагранжа: если функция непрерывна на отрезке и дифференцируема на интервале, найдется точка интервала такая, 
что приращение функции на этом отрезке есть длина отрезка на значение производной функции в этой точке, 

и представим числитель нашей функции в виде:

$$ f(x+t)-f(x+0) = t f^\prime(p_t)$$ 
$$ |t f^\prime(p_t)| < 2k$$

Мы доказали интегрируемость нашей функции, следовательно, по теореме Римана-Лебега первый интеграл стремится к нулю.

Следовательно, и исходная разность $S_n(x) - f(x) \rightarrow 0$, т.е. ряд Фурье сходится к функции $f$. 

 \newpage
\subsection{Дифференицрование и интегрирование рядов Фурье}
  Пусть есть $f(x)$, имеющая только разрывы первого рода.
  
  Тогда она интегрируема.
  
  Тогда ей сопоставляется ряд Фурье.
  
  $$F(x) = \int_{0}^{x} (f(t) - \frac{a_0}{2}) \dd t$$
  
  $$F(0) = 0$$
  
  $$F(2\pi) = \int_{0}^{2\pi} \big( f(t)) - \frac{a_0}{2} \big) \dd t = \frac{a_0}{\pi} - \frac{a_0}{\pi} = 0$$
  
  $F$ - непрерывна, дифференцируема, ограничена, т.е. удовлетворяет всем условиям предыдущей теоремы.
  Тогда:
  
  $$F(x) = \frac{A_0}{2}+ \sum \big( A_n\cos{nx} + B_n\sin{nx} \big)$$
  
  \begin{align*}
      A_n & \stackbin{n > 0}{=} \frac{1}{\pi} \int_{0}^{2\pi} F(t)\cos{nt}\dd t = \frac{1}{\pi} \big( \frac{\sin{nt}}{n} F(t) - \int (f(t) - \frac{a}{2}) \frac{\sin{nt}}{n} \dd t \big) = \\
  &= -\frac{1}{\pi} \int_{0}^{2\pi} f(t)\sin{nt}\dd t = -\frac{b_n}{n} \\ 
  \end{align*}
  Аналогично,
  
  $$ B_n = \frac{a_n}{n}$$
  
  $$F(0) = 0 = A_0 + \sum A_n$$
  
  $$A_0 = 2 \sum_{1}^{\infty} \frac{b_n}{n}$$
  
  \paragraph{} 
  
  Предположим, $\frac{a_0}{2} = 0$.
 
  \begin{align*}
      \int_{a}^{b} f(t)\dd t &= \int_{0}^{b} \big(f(t)-\frac{a_0}{2}\big)\dd t - \int_{0}^{a} \big(f(t) - \frac{a_0}{2} \big) \dd t = F(b) - F(a) = \\
      &= \sum_{n} \frac{a_n}{n}\big( \sin{na} - \sin{nb} \big) - \sum_{n} \frac{b_n}{n} \big( \cos{na} - \cos{nb} \big) \\
  \end{align*}
  
  То есть для непрерывной функции мы можем интегрировать на промежутке ряд Фурье, даже если он не сходится.
  
 
\vspace{1em}

\subsection{Гладкость функции и скорость сходимости её ряда Фурье}

Пусть $f \in C_1(\real)$, тогда $f(x) = $ ряд Фурье.

Рассмотрим $f'$.

Из условия $f \in C_1(\real)$, $f'$ можно сопоставить ряд Фурье:

$$a_0' = 0 \qquad a_n' = n b_n \qquad b_n' = -n a_n$$

Вывод: у гладкой функции $a_n = o(\frac{1}{n}),$ как и $ b_n$ (так как коэффициенты Фурье стремятся к нулю).


Если $f \in C_k(\real)$, то повторяя дифференцирование $k$ раз, получим:

$$a_n = o \big( \frac{1}{n^k} \big) \qquad b_n = o \big( \frac{1}{n^k} \big) $$.

Вывод: чем более гладкая функция, тем быстрее её коэффициенты Фурье стремятся к нулю.

\vline


Вспомним признак Вейрштрасса равномерной сходимости ряда.

Понятно, что  если функция не была непрерывной, то равномерной сходимости нет.

Верно и обратное: если для функции выполнены достаточные условия сходимости ряда Фурье, то на любом отрезке, где функция непрерывна, сходимость ряда Фурье равномерная.

\vspace{2em}

\textbf{Определение.} Явление Гиббса — некоторое особое поведение частичных сумм ряда Фурье в окрестности точки разрыва разлагаемой функции.

А именно: выбросы по амплитуде примерно на 18\%.

При этом для любой разрывной функции будет одно и то же. На этом основаны методы улучшения сходимости рядов Фурье.

\newpage

Рассмотрим частичные суммы ряда Фурье:

$$|f(x) - S_N(x)| = |R_N(x)|$$

Пусть $f$ - $2\pi$-периодическая, непрерывная, дифференцируемая.

$$R_N(x) = |\sum_{N+1}^{\infty} a_n\cos{nx} + b_n\sin{nx}| = \dots $$

$$\sup(|a_n^{(k)}+b_n^{(k)}|) = \gamma_n \rightarrow 0, \qquad \gamma_n \le L$$

$$|a_n| + |b_n| \le \frac{\gamma_n}{n^k}$$

$$ \dots \le \sum_{N+1}^{\infty} |a_n| + |b_n|  \le L \sum_{N+1}^{\infty} \frac{1}{n^k} \le L \int_{N}^{\infty} \frac{1}{x^k} \dd x \le \frac{L}{n^{k-1}}$$

\paragraph{}

Пусть у нас есть ряд Фурье,
$a_n, b_n = O(\frac{1}{n^2})$
Ряд Фурье сходится (т.к. его мажорирует сходящаяся функция), сумма ряда непрерывная функция.

Если 
$a_n, b_n = O(\frac{1}{n^3})$
, то
формально продифференцировав ряд Фурье, получим, что коэффициенты получившегося ряда будут $O(\frac{1}{n^2})$, то есть производная тоже будет непрерывной функцией.

\vspace{1em}

Вывод: если $a_n, b_n = O(\frac{1}{n^{k+2}})$, то сумма ряда $f \in C^K$.

\newpage

\subsection{Равенство Парсеваля-Ляпунова}

Пусть функции $f$ сопоставлен ряд Фурье c коэффициентами($a_n, b_n$), функции g - с коэффициентами($c_n, d_n$) 
$f, g \in C^2(\real) \rightarrow$ ряды сходятся.
Тогда:

$$ \frac{1}{\pi} \int_{-\pi}^{\pi}f(x)g(x) \dd x = \frac{1}{\pi} \int f(x) \big( \frac{c_0}{2} + \sum c_n \cos{nx} + d_n \sin{nx} \big) \dd x 
= c_0 a_0 / 2 + \sum \big( a_n c_n + b_n d_n \big) $$

Полученное нами равенство назывется \textbf{равенство Парсеваля}.

\vspace{1em}

Если подставить в него $f = g$, то получится \textbf{равенство Ляпунова}:

$$\frac{1}{\pi} \int_{-\pi}^{\pi} f^2(t) \dd t = \frac{a_0^2}{2} + \sum \big(a_n^2 + b_n^2 \big)$$.

\vspace{1em}

На самом деле для справедливости равенств достаточно, чтобы $f, g \in L_2(\real)$. 

\subsection{Комплексная форма рядов Фурье}

Пусть функции $f$ сопоставлен ряд Фурье.

Воспользуемся известными формулами:

$$\cos{nx} = \frac{e^{inx} + e^{-inx}}{2} \qquad \sin{nx} = i \frac{-e^{inx} + e^{-inx}}{2}$$

\vspace{1em}

\begin{align*} 
    f & \sim \frac{a_0}{2} + \sum_{1}^{\infty} a_n  \frac{e^{inx} + e^{-inx}}{2} + b_n i \frac{-e^{inx} + e^{-inx}}{2} = \\ 
    & = \frac{a_0}{2} + \sum_{-\infty}^{-1} \frac{a_{-n} + ib_{-n}}{2}e^{inx} + \sum_{1}^{\infty} \frac{a_n-ib_n}{2}e^{inx} = \dots
\end{align*}

Посмотрим, как выглядят коэффициенты

$$\frac{a_n-ib_n}{2} := \gamma_n = \frac{1}{2\pi} \int_{-\pi}^{\pi} f(x) e^{-inx} \dd x \hspace{4em} \frac{a_0}{2} = \frac{1}{2\pi} \int_{-\pi}^{\pi} f(t) \dd t$$

$$ \dots = \sum_{-\infty}^{\infty} \gamma_n e^{ikx}$$

\paragraph{}

Запишем равенство Парсеваля в комплексной форме:

$$ \frac{1}{\pi} \int f \overline{g} \dd x = \sum_{-\infty}^{\infty} \sigma_n \delta_n$$

\subsection{Ядро Фейера}

\[
S_{n}(x)=\frac{1}{\pi}\int(f(x+t)+f(x-t))D_{n}(t)\mathrm{d}t
\]


\[
\tilde{f}(x)=\frac{1}{n}\sum_{0}^{n-1}S_{k}(x)=\frac{1}{\pi}\int_{0}^{\pi}(f(x+t)+f(x-t))\frac{1}{n}\sum D_{k}(t)\mathrm{d}t
\]


\textbf{Определение}.$\frac{1}{n}\sum D_{k}(t)=F_{n}$- ядро Фейера.

Проделаем тригонометрические преобразования:

\[
\frac{1}{n}\sum_{k=1}^{n}\frac{\cos kt-\cos(k+1)t}{4\sin^{2}\frac{t}{2}}=\frac{1}{n}\frac{1-\cos nt}{4\sin^{2}\frac{t}{2}}=\frac{1}{n}\frac{\sin^{2}\frac{n}{2}t}{2\sin^{2}\frac{t}{2}}
\]


\vline

\textbf{Свойства ядра Фейера:}
\begin{enumerate}
\begin{singlespace}
\item Ядро Фейера неотрицательно.\end{singlespace}

\item $\frac{1}{\pi}\int_{0}^{\pi}F_{n}(t)\mathrm{d}t=\frac{1}{2}$
\item Если $t\in(\delta,\pi)$, то $0<F_{n}(t)\le\frac{1}{2n\sin(\delta/2)^{2}}$,
то есть ядро отделено от нуля.
\end{enumerate}
\[
f_{n}(x)=\frac{1}{\pi}\int_{0}^{\pi}(f(x+t)+f(x-t))F_{n}(t)\mathrm{d}t
\]


\begin{doublespace}
\textbf{Теорема} (Фейера). Если функция $f\in C(R)$, $2\pi$-периодическая,
то $\tilde{f}$ сходится к $f$ равномерно на всей оси.
\end{doublespace}

\textbf{Доказательство}.

\begin{align*}{1}
|F_{n}(x)-f(x)|= & |\frac{1}{\pi}\int_{0}^{\pi}((f(x+t)+f(x-t))-2f(x))F_{n}(t)\mathrm{d}t|\le\\
 & |\frac{1}{\pi}\int_{0}^{\delta}((f(x+t)+f(x-t))-2f(x))F_{n}(t)\mathrm{d}t|+|\frac{1}{\pi}\int_{\delta}^{\pi}\dots\mathrm{d}t|\le\\
 & \frac{1}{\pi}\int_{0}^{\delta}(|(f(x+t)+f(x-t))-2f(x)|)F_{n}(t)\mathrm{d}t+\frac{1}{\pi}\int_{\delta}^{\pi}\dots\mathrm{d}t
\end{align*}


\begin{doublespace}
В силу равномерной непрерывности $f$ можно выбрать $\delta$ так,
что $|f(x+t)+f(x-t)|\le\epsilon/2$. \hspace{2em}Значит, первый интеграл
$\le\frac{\epsilon}{2}$.

Функция $f$ ограничена: $|f(x)<M|\rightarrow$$(|(f(x+t)+f(x-t))-2f(x)|)<4M$.
Тогда из третьего свойства ядра Фейера можно выбрать $n$ так, что
второй интеграл $\text{\ensuremath{\le\frac{\epsilon}{2}}}$$\quad\square$
\end{doublespace}

\newpage

\subsection{Теорема Вейрштрасса о равномерном приближении непрерывной функции многочленами}

\begin{doublespace}
Докажем теорему Вейрштрасса о равномерном приближении непрерывной
функции многочленами:
\end{doublespace}

$f\in C[a,b]\rightarrow$$|f(x)-P_{n}(x)|<\epsilon$

\begin{doublespace}
Линейной заменой превратим промежуток в $[0,\pi]$ и чётным образом
продолжим на $[-\pi,0]$

Построим сдвинутую функцию $f_{tr}:$ $\forall x\quad|f_{tr}-F_{\frac{\epsilon}{2}}(x)|<\frac{\epsilon}{2}$.
\end{doublespace}

Для $F_{\frac{e}{2}}$ имеем конечное число слагаемых вида $a_{n}\sin(nx)$.
Будем каждое слагаемое аппроксимировать многочленом с точносью $\frac{\epsilon}{2^{n+1}}$.
Тогда суммарная погрешность приближения функции $F$ будет $\le\frac{\epsilon}{2}$. 

\begin{doublespace}
Значит, наш итоговый многочлен отличается от $f_{tr}$ меньше, чем
на $\epsilon$. $\quad\square$
\end{doublespace}

\newpage
\subsection{Применение рядов Фурье к решению волнового уравнения}

Формулировка задачи:

Есть гибкая упругая струна, закреплённая в точках $0$ и $\pi$.

\[
y(x,t)-?\qquad y(x,0)=h(x)\qquad y_{t}^{'}(x,0)=0
\]


Решение:

\[
h(x)=\sum_{1}^{\infty}b_{n}\sin(nx)
\]


\[
y(x,t)=\sum_{1}^{\infty}b_{n}(t)\sin(nx)
\]


\[
\sum b_{k}^{''}(t)\sin(nx)=-a^{2}\sum b_{n}(t)n^{2}\sin(nx)
\]


\[
b_{n}^{''}(t)+a^{2}n^{2}b_{n}(t)=0
\]
\[
b_{n}(t)=A_{n}\sin(ant)+B_{n}\cos(ant)
\]
\[
\sum b_{n}^{'}(0)\sin(nx)=0
\]
\[
b_{n}^{'}(0)=A_{n}\cos(ant)-B_{n}\sin(ant)=0
\]
\[
A_{n}=0
\]
\[
y(x,t)=\sum b_{n}\cos(ant)\sin(nx)=\sum b_{n}\frac{\sin(n(x+at))+\sin((n(x-at))}{2}=\frac{h(x+at)+h(x-at)}{2}
\]


$h$ - дважды непрерывно дифф. функция, значит, решение найдено.
\end{document}
